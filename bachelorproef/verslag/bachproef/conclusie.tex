%%=============================================================================
%% Conclusie
%%=============================================================================

\chapter{Conclusie}%
\label{ch:conclusie}

Deze scriptie tracht een antwoord te bieden op de volgende onderzoeksvraag:

\begin{itemize}
	\item Hoe kunnen systemen wetenschappelijke artikelen automatisch vereenvoudigen, gericht op de unieke noden van scholieren met dyslexie in de derde graad middelbaar onderwijs?
\end{itemize}

Eerst geeft de requirementsanalyse nieuwe inzichten in huidige toepassingen voor \textit{automatic text simplification} (ATS). Zo ontbreekt personaliseerbaarheid in online tools. Eindgebruikers kunnen geen gepersonaliseerde samenvatting of vereenvoudiging laten maken door deze toepassingen. Verder beschikken tools die enkel lexicale vereenvoudiging toepassen over onvoldoende opmaakopties om de leeservaring van scholieren met dyslexie tijdens het begrijpend lezen van een wetenschappelijk artikel te bevorderen. Daarnaast kunnen eindgebruikers met deze toepassingen geen gepersonaliseerde vereenvoudiging of samenvatting maken. Daartegenover kunnen zij geen wetenschappelijke artikelen inladen in de toepassingen die wel gepersonaliseerde ATS kunnen uitvoeren. 

\medspace

Bing Chat en ChatGPT bieden mogelijkheden voor ATS aan. Echter vereisen deze een uitgebreide informaticakennis, ofwel een vaardigheid waarover de meeste scholieren en leraren niet bezitten. Ontwikkelaars kunnen de achterliggende taalmodellen gebruiken om toepassingen te maken, maar zij richten zich hoofdzakelijk op samenvattingstools. Dat draagt niet noodzakelijk bij tot een eenvoudigere tekst en daardoor komt het leereffect voor scholieren niet ten goede. Huidige toepassingen bewijzen nochtans dat ontwikkelaars dergelijke toepassingen kunnen ontwikkelen. Deze requirementsanalyse benadrukt de noodzaak van een gebruiksvriendelijke toepassing in het onderwijs, waarmee scholieren en leerkrachten wetenschappelijke teksten op een eenvoudige manier kunnen vereenvoudigen.

\medspace

Vervolgens wijst een vergelijking van taalmodellen uit dat HuggingFace (HF) taalmodellen, specifiek getraind op vereenvoudigingsopdrachten, lexicale vereenvoudiging kunnen mogelijk maken. GPT-3, een geavanceerd taalmodel doet dit beter door ook syntactische vereenvoudiging aan te bieden en formaatwijzigingen, ongezien in huidige toepassingen. Zo produceert het ook tekst met minder lange en complexe woorden. Dit taalmodel kan doelgroepen in grote lijn inschatten, waartoe andere tools niet kunnen. Geteste HF taalmodellen genereren minder coherente tekst en kunnen zinnen samensmelten. Daarnaast vereisen zij een extra vertaalfase, wat GPT-3 niet moet doen. Zo moeten ontwikkelaars geen extra vertaalfase ontwikkelen wanneer zij teksten met GPT-3 willen vereenvoudigen. Daarom moet het prototype specifieke prompts en technieken aangegeven door \textcite{McFarland2023, White2023} gebruiken.

\medspace

Tot slot toont de ontwikkeling van Pentimentor aan dat ontwikkelaars ATS-software kunnen ontwikkelen met \textit{open-source} AI en NLP-technologieën. Zo kunnen zij PDFMiner en Layoutparser gebruiken om tekstinhoud uit wetenschappelijke artikelen te extraheren, met of zonder behoud van de oorspronkelijke titelstructuur. Bovendien kunnen ze met OpenAI's GPT-3 API gepersonaliseerde ATS-toepassingen ontwikkelen door geschikte prompts te gebruiken. Vervolgens kunnen zij met Pandoc gepersonaliseerde documenten in docx-formaat automatisch genereren. Ontwikkelaars kunnen basis Javascript toepassen om eenduidige handelingen voor eindgebruikers te onwikkelen, die voordien enkel per commandline mogelijk waren. Zo kunnen zij webpagina's opbouwen die voldoen aan de noden beschreven in \textcite{Rello2012a}.  Hoewel het prototype niet aan alle \textit{should} en \textit{could-haves} voldoet, kunnen ontwikkelaars met de genoemde softwarepakketten een volledig functionele toepassing ontwikkelen.

\medspace

Dit onderzoek benadrukt de beschikbaarheid van complexe taalmodellen en tools voor ontwikkelaars om toepassingen voor ATS te kunnen ontwikkelen. Zo kan GPT-3 als een toepasselijk taalmodel dienen, want het presteert goede resultaten op de toegepaste leesmetrieken, onder andere het aantal complexe en lange woorden per zin. Daarnaast kan dit taalmodel korte annotaties genereren voor vakjargon of onbekende woordenschat voor scholieren. Verder kunnen ontwikkelaars deze doelgroepen parameteriseren en zo verschillende teksten genereren op maat van hun unieke noden. Echter moeten ontwikkelaars rekening houden met de doelgroep waarvoor zij dergelijke toepassingen ontwikkelen, want taalmodellen garanderen geen correcte inschatting. Extra trainingsdata kan het model helpen bij de doelgroepsinschatting, zoals aangeraden door \textcite{Gooding2022} in de vorm van leerstof op leesniveau van de doelgroep.