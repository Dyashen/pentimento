%%=============================================================================
%% Discussie
%%=============================================================================

\chapter{\IfLanguageName{dutch}{Discussie}{Discussie}}%
\label{ch:discussie}

Om te achterhalen hoe ontwikkelaars gepersonaliseerde \textit{automatic text simplification} (ATS) kunnen bieden, gebruikt het onderzoek drie onderzoeksmethoden. Het onderzoek focust enkel op gepersonaliseerde ATS voor scholieren met dyslexie in de derde graad van het middelbaar onderwijs.

\medspace

Allereerst wijzen de resultaten van de requirementsanalyse uit dat zowel erkende toepassingen voor tekstvereenvoudiging als online tools onvoldoende functionaliteiten bieden voor personalisatie naar specifieke behoeften. Dit resultaat komt overeen met de verwachting dat ontwikkelaars hun toepassingen niet richten op gepersonaliseerde ATS. Daarnaast komt het overeen met de verwachtingen dat ontwikkelaars geen rekening houden met dyslectische scholieren in de derde graad van het middelbaar onderwijs. Hiervoor geeft het onderzoek enkele verklaringen: de complexiteit die gepaard gaat met de ontwikkeling van zulke toepassingen, het gebrek aan initiatief binnen het informatica vakgebied en de beperkte populariteit van samenvattingstools, zoals aangegeven door \textcite{Gooding2022} in de literatuurstudie.

\medspace

Hoewel de functionaliteiten van ChatGPT en Bing Chat mogelijk kunnen bijdragen aan gepersonaliseerde ATS, ontbreken deze toepassingen eenduidige handelingen. Dit vormt een obstakel voor gebruikers waardoor zij moeite kunnen ervaren bij het handmatig vereenvoudigen van wetenschappelijke artikelen. Daarnaast houdt ChatGPT geen rekening met verwijzingen of artikelen buiten de getrainde data, wat de credibiliteit van de teruggekregen tekst kan verlagen. Bing Chat vermijdt dit probleem door rekening te houden met externe referenties. Daarmee geeft het systeem, naast de vereenvoudigde tekst, ook direct de aangehaalde referenties mee aan de eindgebruiker. Ontwikkelaars kunnen deze toepassing of dergelijke API gebruiken om nauwkeurige toepassingen met referentiemateriaal aan te bieden in ondersteunende onderwijstoepassingen. Verder onderzoek naar de toepassing van deze AI-technologieën via een API is noodzakelijk en kan baanbrekend zijn voor de onderwijssector, ondersteund door \textcite{Roose2023, Garg2022}. Anderzijds is er de mogelijkheid om bestaande toepassingen, zoals Kurzweil, uit te breiden met functionaliteiten die gepersonaliseerde ATS aanbieden aan scholieren met dyslexie in de derde graad van het middelbaar onderwijs. Tot slot moeten onderzoekers meer experimenten uitvoeren naar het gebruik van Bing Chat en ChatGPT in het onderwijs. 

\medspace

Bij de tweede onderzoeksmethode vergeleek het onderzoek vier verschillende \textit{pre-trained} taalmodellen. Deze onderzoeksfase wijst uit dat ontwikkelaars het GPT-3 taalmodel kunnen gebruiken voor gepersonaliseerde ATS. Het onderzoek gaat niet verder dan het finetunen van de API-parameters. Zo bevat het geen extra tekstdata van wetenschappelijke artikelen. Verder wijst de vergelijkende studie uit dat de drie geteste HuggingFace (HF) taalmodellen en het geteste GPT-3 taalmodel via API beschikken over \textit{Complex Word Identification} (CWI)-functionaliteiten en \textit{substitution generation}. Hoewel de drie HF taalmodellen een gekregen tekst op lexicaal vlak kunnen vereenvoudigen, staan ze in de schaduw van GPT-3. Daarmee kan het GPT-3 taalmodel een baanbrekende oplossing bieden voor gepersonaliseerde ATS van wetenschappelijke artikelen. Het kan snel en efficiënt moeilijke woorden herkennen in lange doorlopende tekst. Tot slot kan het de oorspronkelijke tekst herschrijven als tabel of opsomming, wat huidige toepassingen niet aanbieden.

\medspace

Dit resultaat bevestigt de verwachting dat GPT-3 alle aspecten van gepersonaliseerde ATS kan aanbieden in tegenstelling tot de geteste HF taalmodellen die ze niet allemaal kunnen realiseren. Het onderzoek geeft de complexiteit van de getrainde data als mogelijke verklaring. Zo trainden de onderzoekers de geteste taalmodellen op wetenschappelijke literatuur. Onderzoekers moeten deze verschillen verder onderzoeken. Zo kunnen onderzoekers deze verschillen beter begrijpen, specifiek binnen de context van wetenschappelijke artikelen. Daarnaast kunnen \textit{Large Language Models} of LLM's ook prompts van ontwikkelaars beantwoorden. Deze taalmodellen kunnen verschillende antwoorden genereren, maar het onderzoek wijst uit dat ontwikkelaars de \textit{temperature} parameter kunnen aanpassen om één antwoord per prompt te verkrijgen. Ondanks dat er momenteel onvoldoende inzicht is, zou verder onderzoek zich moeten richten op de verschillen tussen prompts en hun bijhorende \textit{temperature}.

\medspace

Hoewel GPT-4 niet tot het onderzoek behoort, kunnen ontwikkelaars hiermee toepassingen maken. \textcite{Simon2021} geeft aan dat de verschillen tussen deze groei van het aantal parameters weinig effect hebben op de kwaliteit van de tekstdata. Toch beginnen ontwikkelaars snel de sprong te maken naar de volgende iteratie van het GPT-taalmodel die ook meer energie van het systeem vereist. Verder onderzoek moet uitwijzen of de sprong van GPT-3 naar GPT-4 al dan niet merkbaar is op taalvlak. Een opvolgend onderzoek met dit taalmodel is vereist om te testen of het taalmodel over voldoende data beschikt om wetenschappelijke artikelen te vereenvoudigen op maat van scholieren met dyslexie in de derde graad van het middelbaar onderwijs. 

\medspace

Vervolgens kunnen ontwikkelaars hun eigen specifieke prompts opstellen voor deze taalmodellen. Hierin kunnen zij ook de doelgroep meegeven en daarmee een extra parameter meegeven. Dit onderzoek bevestigt niet in welke mate de uitvoer van elkaar verschilt. Deze eenduidige oplossing kan echter een revolutionaire oplossing bieden voor gepersonaliseerde ATS of samenvatting. Daarom moeten onderzoekers hier verder onderzoek op uitvoeren. Verder onderzoek naar doelgroepinschattingen via prompts is ook nodig. Daarnaast zou toekomstig onderzoek zich kunnen richten op het potentieel van de combinatie van GPT-3 met \textit{full-text-search}-technologieën. 

\medspace

Verder stelt het onderzoek de gebruikte machinale beoordeling in vraag. Zo resulteerde de door taalmodellen vereenvoudigde teksten een miniem verschil bij de leesgraadscores FRE en FOG. Daarnaast wijst het experiment uit dat de \textit{readability}-library de actieve stem van een zin niet kan achterhalen. Daarom kan het onderzoek geen vaststelling maken of de taalmodellen van passief naar actief kunnen schrijven. Spacy \textit{word embeddings} kunnen een alternatieve techniek aanreiken om hulpwerkwoorden en vervoegingen van het werkwoord 'zijn' te achterhalen. Daarnaast kunnen toepassingen zoals TextInspector meer metrieken dan de uitgeteste leesgraadscores aanbieden. Daarom moeten onderzoekers verdere experimenten uitvoeren naar de bruikbaarheid van deze libraries. 

\medspace

De derde en laatste onderzoeksfase toont aan hoe ontwikkelaars toepassingen voor gepersonaliseerde ATS kunnen ontwikkelen, specifiek op maat voor scholieren met dyslexie in de derde graad van het middelbaar onderwijs. Zo dienen de verworven kennis en aangeleerde tools uit de richtingen Toegepaste Informatica aan Vlaamse hogescholen als startpunten om \textit{Pentimentor} te ontwikkelen. Pentimentor dient slechts als een haalbaarheidstoetsing voor ontwikkelaars. Het is belangrijk voor de lezer om te beseffen dat de ontwikkeling gebaseerd is op onderzochte kenmerken en technieken die de impact van handmatige tekstvereenvoudiging al hebben uitgewezen. Daarnaast gebeurt de ontwikkeling van Pentimentor met het oog op een eenduidige implementatie van de technieken, die voordien enkel beschikbaar waren via CLI. Zo wijst de ontwikkeling van Pentimentor uit dat ontwikkelaars met vrij beschikbare middelen en API's wel dergelijk toepassingen kunnen maken. Dit resultaat komt overeen met de verwachting dat ontwikkelaars over de nodige tools beschikken. Zo baseert het onderzoek zich op de beschikbaarheid van \textit{open source} tools en python-bibliotheken. Die stellen ontwikkelaars in staat om complexe taken eenvoudig te kunnen reproduceren. Toch moet de lezer ervan bewust gemaakt worden dat het doelpubliek van dyslectische scholieren niet werd opgenomen bij de testfase van Pentimentor. Daarom kan dit onderzoek voornamelijk dienen als een haalbaarheidsmeting voor ontwikkelaars.

\medspace

Tot slot kunnen onderzoekers uit het logopedisch vakgebied Pentimentor gebruiken om onderzoek te voeren naar het leereffect op leesbegrip bij scholieren met dyslexie in de derde graad van het middelbaar onderwijs. Dit stemt overeen met de implicaties waar \textcite{Gooding2022} op wijst in haar onderzoek. Daarnaast moeten onderzoekers uit het onderwijsvakgebied de effecten van dergelijk tools observeren bij leerlingen en leerkrachten in het middelbaar onderwijs. Hoewel het onderzoek naar de bruikbaarheid van deze technologieën in het onderwijs zich in een prille fase bevindt, moeten onderzoekers toch de inzet van deze toepassingen verder onderzoeken. Deze bruikbaarheid kan via een browserextensie, lokale of online toepassing.