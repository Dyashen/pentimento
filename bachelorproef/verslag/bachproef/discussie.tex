%%=============================================================================
%% Discussie
%%=============================================================================

\chapter{\IfLanguageName{dutch}{Discussie}{Discussie}}%
\label{ch:discussie}

Om te achterhalen hoe ontwikkelaars gepersonaliseerde \textit{automatic text simplification} (ATS) kunnen bieden, gebruikt het onderzoek drie onderzoeksmethoden. Het onderzoek focust enkel op ATS voor scholieren met dyslexie in de derde graad van het middelbaar onderwijs.

\medspace

Allereerst wijzen de resultaten van de requirementsanalyse uit dat zowel erkende toepassingen als online tools onvoldoende functionaliteiten bieden voor gepersonaliseerde ATS. Daarnaast bieden deze onvoldoende opmaakopties. Daarmee komt dit resultaat overeen met de verwachting dat ontwikkelaars hun toepassingen niet specifiek richten op ATS. Daarnaast komt het overeen met de verwachtingen dat ontwikkelaars geen rekening houden met scholieren met dyslexie in de derde graad van het middelbaar onderwijs bij deze tools. Hiervoor legt het onderzoek enkele verklaringen, onder andere de complexiteit die gepaard gaat met de ontwikkeling van dergelijk toepassing, het gebrek aan iniatief binnen het informaticavakgebied en de populariteit van samenvattingstools, zoals in de literatuurstudie aangegeven door \textcite{Gooding2022}.

\medspace

Hoewel de functionaliteiten van ChatGPT en Bing Chat mogelijke hulp bij gepersonaliseerde ATS kunnen aanbieden, ontbreken deze toepassingen de eenduidige handelingen waardoor gebruikers moeite kunnen hebben bij het handmatig vereenvoudigen van wetenschappelijke artikelen. Daarnaast houdt ChatGPT geen rekening met verwijzingen of artikelen buiten de getrainde data, wat de credibiliteit van de teruggekregen tekst kan verlagen. Daartegenover doet Bing Chat dat wel. Deze toepassing houdt rekening met externe referenties. Daarmee kunnen eindgebruikers snel teksten ophalen waarbij het systeem de referenties direct aan hen meegeeft. Ontwikkelaars kunnen deze toepassing of dergelijke API gebruiken om nauwkeurige toepassingen m referentiemateriaal aan te bieden in ondersteunende onderwijstoepassingen. Verder onderzoek naar de toepassing van deze AI via een API is noodzakelijk en kan baanbrekend zijn voor de onderwijssector, ondersteund door \textcite{Roose2023, Garg2022}. Anderzijds is er de mogelijkheid om bestaande toepassingen, zoals Kurzweil, uit te breiden met functionaliteiten die gepersonaliseerde ATS aanbieden aan scholieren met dyslexie in de derde graad van het middelbaar onderwijs. Tot slot moeten onderzoekers meer experimenten uitvoeren naar het gebruik van Bing Chat en ChatGPT in het onderwijs. 

\medspace

Bij de tweede onderzoeksmethode vergeleek het onderzoek vier verschillende \textit{pre-trained} taalmodellen. Deze onderzoeksfase wijst uit dat ontwikkelaars het GPT-3 taalmodel kunnen gebruiken voor gepersonaliseerde ATS. Zo maakte het geteste GPT-3 model de davinci-engine. Het onderzoek gaat niet verder dan het finetunen van de API-parameters. Zo bevat het geen extra tekstdata van wetenschappelijke artikelen. Verder wijst de vergelijkende studie uit dat de drie geteste HuggingFace (HF) taalmodellen en het geteste GPT-3 taalmodel via API beschikken over \textit{Complex Word Identification} (CWI)-functionaliteiten en \textit{substitution generation}. Hoewel de drie HF taalmodellen een gekregen tekst op lexicaal vlak kunnen vereenvoudigen, toch staan ze in de schaduw van GPT-3. Daarmee kan het GPT-3 taalmodel een baanbrekende oplossing bieden voor gepersonaliseerde ATS van wetenschappelijke artikelen, want het kan snel en efficiënt moeilijke woorden herkennen in doorlopende tekst. Tot slot kan het de oorspronkelijke tekst herschrijven als tabel of opsomming, wat huidige toepassingen niet aanbieden.

\medspace

Dit resultaat bevestigt de verwachting dat GPT-3 alle aspecten van gepersonaliseerde ATS kan aanbieden in tegenstelling tot de geteste HF taalmodellen die deze niet allemaal kan realiseren. Het onderzoek geeft de complexiteit van de getrainde data als mogelijke verklaring. Zo trainden de onderzoekers van de vier geteste taalmodellen deze specifiek op data van wetenschappelijke artikelen. Onderzoekers moeten deze verschillen verder onderzoeken. Zo kunnen zij de verschillen tussen de verschillende taalmodellen beter begrijpen, specifiek binnen de context van wetenschappelijke artikelen. Daarnaast kunnen \textit{Large Language Models} of LLM's ook prompts van ontwikkelaars beantwoorden. Deze taalmodellen kunnen verschillende antwoorden genereren, maar het onderzoek wijst uit dat ontwikkelaars de \textit{temperature} parameter kunnen aanpassen om één antwoord per prompt te verkrijgen. Hierover bestaat er onvoldoende onderzoek en daarom moet verder onderzoek zich richten op de verschillen tussen prompts en hun bijhorende \textit{temperature}. 

\medspace

Hoewel GPT-4 niet tot het onderzoek behoort, toch kunnen ontwikkelaars hiermee toepassingen maken. \textcite{Simon2021} geeft aan dat de verschillen tussen deze groei van het aantal parameters weinig effect hebben op de kwaliteit van de tekstdata. Toch beginnen ontwikkelaars snel de sprong te maken naar de volgende iteratie die ook meer energie van het systeem vereist. Verder onderzoek moet uitwijzen of de sprong van GPT-3 naar GPT-4 al dan niet merkbaar is op taalvlak. Een opvolgend onderzoek met dit taalmodel is vereist om te testen of dit taalmodel over voldoende data beschikt om wetenschappelijke artikelen te vereenvoudigen op maat van scholieren met dyslexie in de derde graad van het middelbaar onderwijs. 

\medspace

Vervolgens kunnen ontwikkelaars hun eigen specifieke prompts opstellen voor deze taalmodellen. Hierin kunnen zij ook de doelgroep meegeven en daarmee een extra parameter meegeven. Dit onderzoek bevestigt niet de mate waarin deze uitvoer verschilt van elkaar. Deze eenduidige oplossing kan echter een revolutionaire oplossing bieden voor gepersonaliseerde ATS of samenvatting. Daarom moeten onderzoekers hier verder onderzoek op uitvoeren. Verder onderzoek naar doelgroepinschattingen via prompts is ook nodig. Daarnaast zou toekomstig onderzoek zich kunnen richten op het potentieel van de combinatie van GPT-3 en textit{full-text-search}-technologieën. 

\medspace

Verder stelt het onderzoek de gebruikte machinale beoordeling in vraag. Zo resulteerde de vereenvoudigde teksten door taalmodellen een miniem verschil bij de leesgraadscores FRE en FOG. Daarnaast wijst het experiment uit dat de \textit{readability}-library de actieve stem van een zin niet kan achterhalen. Daarom kan het onderzoek geen vaststelling maken of de taalmodellen van passief naar actief kunnen schrijven. Spacy \textit{word embeddings} kunnen een alternatieve techniek aanreiken om hulpwerkwoorden en vervoegingen van het werkwoord 'zijn' te achterhalen. Daarnaast kunnen toepassingen zoals TextInspector meer metrieken dan de uitgeteste leesgraadscores aanbieden. Daarom moeten onderzoekers verdere experimenten uitvoeren naar de bruikbaarheid van deze libraries. 

\medspace

De derde en laatste onderzoeksfase toont aan hoe ontwikkelaars toepassingen voor gepersonaliseerde ATS kunnen ontwikkelen. Dit specifiek op maat voor scholieren met dyslexie in de derde graad van het middelbaar onderwijs. Zo dienen de verworven kennis en aangeleerde tools uit de richtingen Toegepaste Informatica aan Vlaamse Hogescholen als startpunten om \textit{Pentimentor} te ontwikkelen. Hoewel Pentimentor slechts dient als een haalbaarheidstoetsing voor ontwikkelaars, toch moet de lezer rekening houden dat de ontwikkeling zich baseert op onderzochte kenmerken en technieken die de impact van handmatige tekstvereenvoudiging al hebben uitgewezen. Daarnaast gebeurt de ontwikkeling van Pentimentor met het oog op een eenduidige implementatie van technieken, die voordien enkel beschikbaar waren via CLI. Zo wijst de ontwikkeling van Pentimentor uit dat ontwikkelaars met vrij beschikbare middelen en API's wel dergelijk toepassingen kunnen maken.

\medspace

Dit resultaat komt overeen met de verwachting dat ontwikkelaars over de benodigde tools beschikken om dergelijk prototypen te maken. Hiervoor baseert het onderzoek zich op de beschikbaarheid van textit{open-source} tools en python-bibliotheken. Die stellen ontwikkelaars in staat om complexe taken eenvoudig te kunnen reproduceren. Hoewel Pentimentor aan alle vooropgestelde \textit{must-haves} voldoet, toch moet de lezer zich ervan bewust maken dat het doelpubliek niet werd opgenomen bij de testfase van Pentimentor. Daarom kan het enkel dienen als een haalbaarheidsmeting voor ontwikkelaars.

\medspace

Tot slot kunnen onderzoekers uit het logopedisch vakgebied Pentimentor gebruiken om onderzoek uit te voeren naar het leereffect op leesbegrip bij scholieren met dyslexie in de derde graad van het middelbaar onderwijs. Dit stemt overeen met de implicaties waar \textcite{Gooding2022} op wijst in haar onderzoek. Daarnaast moeten onderzoekers uit het onderwijsvakgebied de effecten van dergelijk tools observeren bij leerlingen en leerkrachten in het middelbaar onderwijs. Hoewel het onderzoek naar de bruikbaarheid van deze technologieën in het onderwijs zich in een prille fase bevindt, toch moeten onderzoekers de inzet van deze toepassingen verder onderzoeken. Deze bruikbaarheid kan via een browserextensie, lokale of online toepassing.