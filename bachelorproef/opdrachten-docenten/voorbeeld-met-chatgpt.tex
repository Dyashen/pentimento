\documentclass{report}
\title{Bachelorproef: richtlijnen rond AI-tekstvereenvoudiging}
\author{Dylan Cluyse}
\documentclass[a4paper,12pt,twoside]{report}

\usepackage[margin=1in]{geometry}
\usepackage{amsmath}
\usepackage{titling}
\usepackage{fontspec}
\usepackage{listings}
\usepackage{hyperref}
\usepackage{graphicx}
\newfontfamily\headingfont[]{Montserrat-Black}
\usepackage[dutch]{babel}
\usepackage{fontspec}
\setmainfont{Montserrat}

\begin{document}
\chapter{Oorspronkelijk artikel}
\section{'Algoritmes kunnen ons ook béter informeren: laten we daar in Vlaanderen werk van maken'}

Netflix lanceerde enkele weken geleden The Social Dilemma, een nieuwe documentaire die de gevaren van sociale media zoals Facebook, TikTok of YouTube blootlegt. De documentaire schetst daarbij een apocalyptisch beeld van onze huidige digitale samenleving die steeds meer gepolariseerd raakt door de algoritmes die achter grote technologieplatformen schuilgaan. Hoewel er gevaren zijn, moeten we ons niet massaal beginnen uitschrijven van sociale media of andere technologieplatformen. Wetenschappelijk onderzoek toont immers aan dat algoritmes ons niet meteen in een informatiebubbel doen belanden. Bovendien kunnen algoritmes ook op zo’n manier ontwikkeld worden dat ze ons net béter kunnen informeren. Laten we daar, op de vooravond van een nieuw parlementair jaar, werk van maken in Vlaanderen.

\subsection{Informatiebubbel}


Het lijkt ironisch, maar Netflix raadt sinds kort zijn abonnees aan om een documentaire te bekijken waarin net wordt gewaarschuwd voor de gevaren van dergelijke aanbevelings- en personalisatiesystemen. In The Social Dilemma lichten verschillende ex-werknemers van Google, Facebook en Twitter toe hoe ze jarenlang hun schouders hebben gezet onder het ontwikkelen van systemen met als enige doel jouw aandacht te trekken en vast te houden. Daarbij leggen ze uit dat deze algoritmes gebruikers leiden naar een informatiebubbel van gekleurd nieuws – al dan niet fake – die er vervolgens voor zorgt dat verschillende groepen in onze samenleving elk hun eigen setje van feiten hebben en hierdoor gepolariseerd raken.


De gevolgen van deze personalisatiesystemen zijn, volgens de makers van The Social Dilemma, niet op één hand te tellen. De algoritmes die ons op persoonlijke basis video’s en nieuwsartikelen aanbevelen, zouden aan de basis liggen van onder meer smartphoneverslavingen, ondemocratische verkiezingsuitslagen en een reeks van zelfmoorden bij tienermeisjes. In De Standaard trekt men zelfs de parallel met de klimaatcrisis waarbij sociale media een minstens even grote bedreiging vormt voor ons existentieel bestaan dan de opwarming van de aarde.


\subsection{Wat zegt de wetenschap?}

Hoewel er iets te zeggen valt over de manier waarop Facebook en co hun algoritmes ontwerpen, is er op dit moment geen sluitend bewijs voor het bestaan van zogenaamde informatiebubbels waarin we niet langer blootgesteld worden aan diverse ideeën of perspectieven. Verschillende studies tonen aan dat veel mensen nog steeds verschillende kanalen gebruiken om op de hoogte te blijven van de actualiteit (zie bijvoorbeeld deze Nederlandse studie). Daarnaast laten ook verschillende studies zien dat mensen nog steeds op regelmatige basis ‘toevallig nieuws’ tegenkomen (zie bijvoorbeeld deze Vlaams studie). Naast het gepersonaliseerde nieuwsoverzicht van Facebook, Twitter of Google maken velen dus nog steeds gebruik van niet-gepersonaliseerde mediakanalen zoals televisie, radio of krant. Kanalen waar professionele journalisten het nieuwsaanbod samenstellen en die mensen regelmatig in contact brengen met verschillende onderwerpen en meningen.


Toch begrijpen we de oproep om voorzichtig(er) om te springen met de evolutie waarin nieuws steeds meer op maat aan de man wordt gebracht. Hoe meer (nieuws)kanalen gepersonaliseerd worden, hoe minder kans we lopen om toevallig op een nieuwsartikel te botsen dat inhoudelijk niet in lijn ligt met onze eigen interesses, voorkeuren of ideeën. Daarom is het belangrijk om deze evolutie te blijven
opvolgen en na te gaan hoe het fundamentele element van dit verhaal – diversiteit – structureel verankerd kan worden in onze digitale samenleving. Een samenleving die in de toekomst ongetwijfeld nog vele personalisatiesystemen zal voortbrengen.

\subsection{Diversiteitsalgoritme}

Eén van de pistes die hiertoe gevolgd kan worden, is het idee van een zogenaamd diversiteitsalgoritme waarbij diversiteit bevorderen het primaire doeleinde is. In tegenstelling tot commerciële algoritmes die doorgaans meer van hetzelfde aanbieden, tracht een diversiteitsalgoritme net op zoek te gaan naar nieuwsartikelen die inhoudelijk verschillen van wat je al hebt gelezen. Dat kan een artikel zijn waarin de stem van een andere politieke partij wordt verkondigd, maar evengoed een artikel waarin een ander standpunt wordt uitgelegd of een ander onderwerp wordt behandeld. Op die manier krijg je, op structurele wijze, een divers gamma aan nieuwsartikelen op je persoonlijk bord.


In Nederland en Engeland kondigden verschillende mediaorganisaties al aan dat ze aan de slag gaan met dergelijke diversiteitsalgoritmes. Ook aan de Universiteit Gent werd er in het kader van het onderzoeksproject #NewsDNA al enkele jaren onderzoek uitgevoerd naar de ontwikkeling van een diversiteitsalgoritme. Met een werkend prototype toont het project aan dat dit geen zoete droom is, maar een waardig alternatief dat binnen handbereik ligt.

\subsection{Maatschappelijk Relance Comité (MRC)}

We kunnen hopen dat in Vlaanderen nieuwsorganisaties de noodzaak van zo’n diversiteitsalgoritme inzien en de handschoen willen opnemen om deze problematiek niet te laten escaleren. Toch is het ook vooral kijken naar wat de overheid doet. In het rapport dat door het Maatschappelijke Relance Comité (MRC) werd opgesteld om Vlaanderen terug op de rails te krijgen na de coronacrisis, werd alvast een duidelijk signaal gegeven aan minister voor media Benjamin Dalle om een meer sturende rol op te nemen: “Er is nood aan een breed samenwerkingsprogramma met verschillende mediapartijen, zowel m.b.t. innovatie als implementatie, ten aanzien van de betrouwbaarheid, diversiteit en vindbaarheid van informatie. Zorg voor permanente validatie en monitoring van de resultaten“ (p. 70). Met de Septemberverklaring in aantocht, onderschrijven we graag die oproep: maak werk van samenwerkingen over mediagroepen heen om een diversiteitsalgoritme te realiseren in Vlaanderen. Op die manier kunnen we, naast de sociale bubbel van vijf, ook onze burgers vrijwaren van een mogelijke tweede bubbel, de informatiebubbel.

Dit opiniestuk werd onderschreven door alle leden van het onderzoeksproject #NewsDNA.

Glen Joris Onderzoeker mict-UGent

Dit opiniestuk verscheen op 28 september in Knack.

\chapter{title}
	
\end{document}